% Options for packages loaded elsewhere
\PassOptionsToPackage{unicode}{hyperref}
\PassOptionsToPackage{hyphens}{url}
\PassOptionsToPackage{dvipsnames,svgnames,x11names}{xcolor}
%
\documentclass[
  a4paper,
  DIV=11,
  numbers=noendperiod]{scrartcl}

\usepackage{amsmath,amssymb}
\usepackage{iftex}
\ifPDFTeX
  \usepackage[T1]{fontenc}
  \usepackage[utf8]{inputenc}
  \usepackage{textcomp} % provide euro and other symbols
\else % if luatex or xetex
  \usepackage{unicode-math}
  \defaultfontfeatures{Scale=MatchLowercase}
  \defaultfontfeatures[\rmfamily]{Ligatures=TeX,Scale=1}
\fi
\usepackage{lmodern}
\ifPDFTeX\else  
    % xetex/luatex font selection
\fi
% Use upquote if available, for straight quotes in verbatim environments
\IfFileExists{upquote.sty}{\usepackage{upquote}}{}
\IfFileExists{microtype.sty}{% use microtype if available
  \usepackage[]{microtype}
  \UseMicrotypeSet[protrusion]{basicmath} % disable protrusion for tt fonts
}{}
\makeatletter
\@ifundefined{KOMAClassName}{% if non-KOMA class
  \IfFileExists{parskip.sty}{%
    \usepackage{parskip}
  }{% else
    \setlength{\parindent}{0pt}
    \setlength{\parskip}{6pt plus 2pt minus 1pt}}
}{% if KOMA class
  \KOMAoptions{parskip=half}}
\makeatother
\usepackage{xcolor}
\setlength{\emergencystretch}{3em} % prevent overfull lines
\setcounter{secnumdepth}{5}
% Make \paragraph and \subparagraph free-standing
\makeatletter
\ifx\paragraph\undefined\else
  \let\oldparagraph\paragraph
  \renewcommand{\paragraph}{
    \@ifstar
      \xxxParagraphStar
      \xxxParagraphNoStar
  }
  \newcommand{\xxxParagraphStar}[1]{\oldparagraph*{#1}\mbox{}}
  \newcommand{\xxxParagraphNoStar}[1]{\oldparagraph{#1}\mbox{}}
\fi
\ifx\subparagraph\undefined\else
  \let\oldsubparagraph\subparagraph
  \renewcommand{\subparagraph}{
    \@ifstar
      \xxxSubParagraphStar
      \xxxSubParagraphNoStar
  }
  \newcommand{\xxxSubParagraphStar}[1]{\oldsubparagraph*{#1}\mbox{}}
  \newcommand{\xxxSubParagraphNoStar}[1]{\oldsubparagraph{#1}\mbox{}}
\fi
\makeatother


\providecommand{\tightlist}{%
  \setlength{\itemsep}{0pt}\setlength{\parskip}{0pt}}\usepackage{longtable,booktabs,array}
\usepackage{calc} % for calculating minipage widths
% Correct order of tables after \paragraph or \subparagraph
\usepackage{etoolbox}
\makeatletter
\patchcmd\longtable{\par}{\if@noskipsec\mbox{}\fi\par}{}{}
\makeatother
% Allow footnotes in longtable head/foot
\IfFileExists{footnotehyper.sty}{\usepackage{footnotehyper}}{\usepackage{footnote}}
\makesavenoteenv{longtable}
\usepackage{graphicx}
\makeatletter
\newsavebox\pandoc@box
\newcommand*\pandocbounded[1]{% scales image to fit in text height/width
  \sbox\pandoc@box{#1}%
  \Gscale@div\@tempa{\textheight}{\dimexpr\ht\pandoc@box+\dp\pandoc@box\relax}%
  \Gscale@div\@tempb{\linewidth}{\wd\pandoc@box}%
  \ifdim\@tempb\p@<\@tempa\p@\let\@tempa\@tempb\fi% select the smaller of both
  \ifdim\@tempa\p@<\p@\scalebox{\@tempa}{\usebox\pandoc@box}%
  \else\usebox{\pandoc@box}%
  \fi%
}
% Set default figure placement to htbp
\def\fps@figure{htbp}
\makeatother

\usepackage{booktabs}
\usepackage{caption}
\usepackage{longtable}
\usepackage{colortbl}
\usepackage{array}
\usepackage{anyfontsize}
\usepackage{multirow}
\usepackage{float}
\usepackage{tabularray}
\usepackage[normalem]{ulem}
\usepackage{graphicx}
\UseTblrLibrary{booktabs}
\UseTblrLibrary{rotating}
\UseTblrLibrary{siunitx}
\NewTableCommand{\tinytableDefineColor}[3]{\definecolor{#1}{#2}{#3}}
\newcommand{\tinytableTabularrayUnderline}[1]{\underline{#1}}
\newcommand{\tinytableTabularrayStrikeout}[1]{\sout{#1}}
\KOMAoption{captions}{tableheading}
\makeatletter
\@ifpackageloaded{caption}{}{\usepackage{caption}}
\AtBeginDocument{%
\ifdefined\contentsname
  \renewcommand*\contentsname{Inhaltsverzeichnis}
\else
  \newcommand\contentsname{Inhaltsverzeichnis}
\fi
\ifdefined\listfigurename
  \renewcommand*\listfigurename{Abbildungsverzeichnis}
\else
  \newcommand\listfigurename{Abbildungsverzeichnis}
\fi
\ifdefined\listtablename
  \renewcommand*\listtablename{Tabellenverzeichnis}
\else
  \newcommand\listtablename{Tabellenverzeichnis}
\fi
\ifdefined\figurename
  \renewcommand*\figurename{Abbildung}
\else
  \newcommand\figurename{Abbildung}
\fi
\ifdefined\tablename
  \renewcommand*\tablename{Tabelle}
\else
  \newcommand\tablename{Tabelle}
\fi
}
\@ifpackageloaded{float}{}{\usepackage{float}}
\floatstyle{ruled}
\@ifundefined{c@chapter}{\newfloat{codelisting}{h}{lop}}{\newfloat{codelisting}{h}{lop}[chapter]}
\floatname{codelisting}{Listing}
\newcommand*\listoflistings{\listof{codelisting}{Listingverzeichnis}}
\makeatother
\makeatletter
\makeatother
\makeatletter
\@ifpackageloaded{caption}{}{\usepackage{caption}}
\@ifpackageloaded{subcaption}{}{\usepackage{subcaption}}
\makeatother

\ifLuaTeX
\usepackage[bidi=basic]{babel}
\else
\usepackage[bidi=default]{babel}
\fi
\babelprovide[main,import]{nswissgerman}
% get rid of language-specific shorthands (see #6817):
\let\LanguageShortHands\languageshorthands
\def\languageshorthands#1{}
\ifLuaTeX
  \usepackage[german]{selnolig} % disable illegal ligatures
\fi
\usepackage{bookmark}

\IfFileExists{xurl.sty}{\usepackage{xurl}}{} % add URL line breaks if available
\urlstyle{same} % disable monospaced font for URLs
\hypersetup{
  pdftitle={Tidycomm-tests},
  pdfauthor={Test Bär},
  pdflang={de-CH},
  colorlinks=true,
  linkcolor={blue},
  filecolor={Maroon},
  citecolor={Blue},
  urlcolor={Blue},
  pdfcreator={LaTeX via pandoc}}


\title{Tidycomm-tests}
\author{Test Bär}
\date{}

\begin{document}
\maketitle

\renewcommand*\contentsname{Inhaltsverzeichnis}
{
\hypersetup{linkcolor=}
\setcounter{tocdepth}{3}
\tableofcontents
}

\section{Regressionsanalyse mit den Daten ``World of
Journalism''}\label{regressionsanalyse-mit-den-daten-world-of-journalism}

Es ist immer ratsam sich zunächst die Regressionskoeffizienten genau
anzuschauen, was mit einer Tabelle praktisch am besten geht, wie sie in
Tabelle~\ref{tbl-tab1} einsehbar ist.

\begin{table}

\caption{\label{tbl-tab1}Regression auf autonome Auswahl}

\centering{

\fontsize{12.0pt}{14.4pt}\selectfont
\begin{tabular*}{0.85\linewidth}{@{\extracolsep{\fill}}lrrrrrrr}
\toprule
 & \multicolumn{4}{c}{unstd.} & std. & \multicolumn{2}{c}{sig.} \\ 
\cmidrule(lr){2-5} \cmidrule(lr){6-6} \cmidrule(lr){7-8}
Variable & B & SE B & LL & UL & B* & t & p \\ 
\midrule\addlinespace[2.5pt]
(Intercept) & 3.66 & 0.13 & 3.41 & 3.90 & — & 29.01 & <.001 \\ 
work\_experience & 0.01 & 0.00 & 0.01 & 0.02 & .160 & 5.37 & <.001 \\ 
trust\_government & 0.04 & 0.03 & -0.01 & 0.09 & .040 & 1.50 & .130 \\ 
ethics\_1 & -0.04 & 0.03 & -0.09 & 0.01 & -.040 & -1.43 & .150 \\ 
ethics\_2 & 0.01 & 0.02 & -0.03 & 0.06 & .020 & 0.68 & .490 \\ 
ethics\_3 & 0.00 & 0.02 & -0.05 & 0.04 & -.000 & -0.06 & .950 \\ 
ethics\_4 & -0.03 & 0.02 & -0.07 & 0.01 & -.050 & -1.46 & .140 \\ 
\bottomrule
\end{tabular*}
\begin{minipage}{\linewidth}
Autonomy Selection, R² = .034\\
F(6,1177) = 7, p = < .001, CI-Level = 95\%\\
\end{minipage}

}

\end{table}%

\begin{center}
\includegraphics[width=0.85\linewidth,height=\textheight,keepaspectratio]{testthat_files/figure-pdf/unnamed-chunk-4-1.pdf}
\end{center}

\subsection{Teiltabelle}\label{teiltabelle}

\subsection{Analyse der
Voraussetzungen}\label{analyse-der-voraussetzungen}

In Abbildung~\ref{fig-reslev} ist gut zu erkennen.

\begin{figure}[H]

\centering{

\centering{

\includegraphics[width=0.85\linewidth,height=\textheight,keepaspectratio]{testthat_files/figure-pdf/fig-reslev-1.pdf}

}

\subcaption{\label{fig-reslev}}

}

\caption{\label{fig-reslev}residualsleverage plot}

\end{figure}%

Schaut man sich darüber hinaus Abbildung~\ref{fig-scaleloc} im schönen
UZH-Design an, wird einem alles klar.

\begin{figure}[H]

\centering{

\centering{

\includegraphics[width=0.85\linewidth,height=\textheight,keepaspectratio]{testthat_files/figure-pdf/fig-scaleloc-1.pdf}

}

\subcaption{\label{fig-scaleloc}}

}

\caption{\label{fig-scaleloc}scalelocation plot}

\end{figure}%

Nicht zuletzt sollte man sich die Residuen in Abhängigkeit der
geschätzten Werte ansehen, was im schönen Viridis-Design in
Abbildung~\ref{fig-resfit} durchaus möglich ist, auch wenn das dunkle
Lila nicht gut zu erkennen ist.

\begin{figure}[H]

\centering{

\centering{

\includegraphics[width=0.85\linewidth,height=\textheight,keepaspectratio]{testthat_files/figure-pdf/fig-resfit-1.pdf}

}

\subcaption{\label{fig-resfit}}

}

\caption{\label{fig-resfit}residualsleverage plot}

\end{figure}%

\section{Häufigkeitsauszählung}\label{huxe4ufigkeitsauszuxe4hlung}

\begin{table}
\fontsize{13.5pt}{16.2pt}\selectfont
\begin{tabular*}{\linewidth}{@{\extracolsep{\fill}}lrrrrr}
\toprule
country & N & Prozent & Valide \% & Kum n & Kum \% \\ 
\midrule\addlinespace[2.5pt]
Austria & 207 & 17\% & 17\% & 207 & 17\% \\ 
Denmark & 376 & 31\% & 31\% & 583 & 49\% \\ 
Germany & 173 & 14\% & 14\% & 756 & 63\% \\ 
Switzerland & 233 & 19\% & 19\% & 989 & 82\% \\ 
UK & 211 & 18\% & 18\% & 1200 & 100\% \\ 
{\textcolor[HTML]{999999}{—}} & {\textcolor[HTML]{999999}{1200}} & {\textcolor[HTML]{999999}{100\%}} & {\textcolor[HTML]{999999}{---}} & {\textcolor[HTML]{999999}{—}} & {\textcolor[HTML]{999999}{---}} \\ 
\bottomrule
\end{tabular*}
\end{table}

\ref{tbl-long-thing} will only appear in PDF

\begin{longtblr}[         %% tabularray outer open
caption={A long table \label{tbl-long-thing}},
]                     %% tabularray outer close
{                     %% tabularray inner open
width={0.7\linewidth},
colspec={X[]X[]X[]X[]X[]X[]},
}                     %% tabularray inner close
\toprule
mpg & cyl & disp & hp & drat & wt \\ \midrule %% TinyTableHeader
21.0 & 6 & 160.0 & 110 & 3.90 & 2.620 \\
21.0 & 6 & 160.0 & 110 & 3.90 & 2.875 \\
22.8 & 4 & 108.0 &  93 & 3.85 & 2.320 \\
21.4 & 6 & 258.0 & 110 & 3.08 & 3.215 \\
18.7 & 8 & 360.0 & 175 & 3.15 & 3.440 \\
18.1 & 6 & 225.0 & 105 & 2.76 & 3.460 \\
14.3 & 8 & 360.0 & 245 & 3.21 & 3.570 \\
24.4 & 4 & 146.7 &  62 & 3.69 & 3.190 \\
22.8 & 4 & 140.8 &  95 & 3.92 & 3.150 \\
19.2 & 6 & 167.6 & 123 & 3.92 & 3.440 \\
17.8 & 6 & 167.6 & 123 & 3.92 & 3.440 \\
16.4 & 8 & 275.8 & 180 & 3.07 & 4.070 \\
17.3 & 8 & 275.8 & 180 & 3.07 & 3.730 \\
15.2 & 8 & 275.8 & 180 & 3.07 & 3.780 \\
10.4 & 8 & 472.0 & 205 & 2.93 & 5.250 \\
10.4 & 8 & 460.0 & 215 & 3.00 & 5.424 \\
14.7 & 8 & 440.0 & 230 & 3.23 & 5.345 \\
32.4 & 4 &  78.7 &  66 & 4.08 & 2.200 \\
30.4 & 4 &  75.7 &  52 & 4.93 & 1.615 \\
33.9 & 4 &  71.1 &  65 & 4.22 & 1.835 \\
21.5 & 4 & 120.1 &  97 & 3.70 & 2.465 \\
15.5 & 8 & 318.0 & 150 & 2.76 & 3.520 \\
15.2 & 8 & 304.0 & 150 & 3.15 & 3.435 \\
13.3 & 8 & 350.0 & 245 & 3.73 & 3.840 \\
19.2 & 8 & 400.0 & 175 & 3.08 & 3.845 \\
27.3 & 4 &  79.0 &  66 & 4.08 & 1.935 \\
26.0 & 4 & 120.3 &  91 & 4.43 & 2.140 \\
30.4 & 4 &  95.1 & 113 & 3.77 & 1.513 \\
15.8 & 8 & 351.0 & 264 & 4.22 & 3.170 \\
19.7 & 6 & 145.0 & 175 & 3.62 & 2.770 \\
15.0 & 8 & 301.0 & 335 & 3.54 & 3.570 \\
21.4 & 4 & 121.0 & 109 & 4.11 & 2.780 \\
21.0 & 6 & 160.0 & 110 & 3.90 & 2.620 \\
21.0 & 6 & 160.0 & 110 & 3.90 & 2.875 \\
22.8 & 4 & 108.0 &  93 & 3.85 & 2.320 \\
21.4 & 6 & 258.0 & 110 & 3.08 & 3.215 \\
18.7 & 8 & 360.0 & 175 & 3.15 & 3.440 \\
18.1 & 6 & 225.0 & 105 & 2.76 & 3.460 \\
14.3 & 8 & 360.0 & 245 & 3.21 & 3.570 \\
24.4 & 4 & 146.7 &  62 & 3.69 & 3.190 \\
22.8 & 4 & 140.8 &  95 & 3.92 & 3.150 \\
19.2 & 6 & 167.6 & 123 & 3.92 & 3.440 \\
17.8 & 6 & 167.6 & 123 & 3.92 & 3.440 \\
16.4 & 8 & 275.8 & 180 & 3.07 & 4.070 \\
17.3 & 8 & 275.8 & 180 & 3.07 & 3.730 \\
15.2 & 8 & 275.8 & 180 & 3.07 & 3.780 \\
10.4 & 8 & 472.0 & 205 & 2.93 & 5.250 \\
10.4 & 8 & 460.0 & 215 & 3.00 & 5.424 \\
14.7 & 8 & 440.0 & 230 & 3.23 & 5.345 \\
32.4 & 4 &  78.7 &  66 & 4.08 & 2.200 \\
30.4 & 4 &  75.7 &  52 & 4.93 & 1.615 \\
33.9 & 4 &  71.1 &  65 & 4.22 & 1.835 \\
21.5 & 4 & 120.1 &  97 & 3.70 & 2.465 \\
15.5 & 8 & 318.0 & 150 & 2.76 & 3.520 \\
15.2 & 8 & 304.0 & 150 & 3.15 & 3.435 \\
13.3 & 8 & 350.0 & 245 & 3.73 & 3.840 \\
19.2 & 8 & 400.0 & 175 & 3.08 & 3.845 \\
27.3 & 4 &  79.0 &  66 & 4.08 & 1.935 \\
26.0 & 4 & 120.3 &  91 & 4.43 & 2.140 \\
30.4 & 4 &  95.1 & 113 & 3.77 & 1.513 \\
15.8 & 8 & 351.0 & 264 & 4.22 & 3.170 \\
19.7 & 6 & 145.0 & 175 & 3.62 & 2.770 \\
15.0 & 8 & 301.0 & 335 & 3.54 & 3.570 \\
21.4 & 4 & 121.0 & 109 & 4.11 & 2.780 \\
\bottomrule
\end{longtblr}




\end{document}
